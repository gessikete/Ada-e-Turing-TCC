%%%%%%%%%%%%%%%%%%%%%%%%%%%%%%%%%%%%%%%%
% Classe do documento
%%%%%%%%%%%%%%%%%%%%%%%%%%%%%%%%%%%%%%%%

% Opções:
%  - Graduação: bacharelado|engenharia|licenciatura
%  - Pós-graduação: [qualificacao], mestrado|doutorado, ppca|ppginf

\documentclass[bacharelado]{UnB-CIC}%

\setcounter{tocdepth}{4}
\setcounter{secnumdepth}{4}

\renewenvironment{quote}
  {\small\list{}{\leftmargin=4cm}%
   \item\relax}
  {\endlist}

\usepackage{url}%
\usepackage{pdfpages}% incluir PDFs, usado no apêndice
\usepackage{multirow}% merge em células da tabela
\usepackage{subfig}%
\usepackage{float}%

%%%%%%%%%%%%%%%%%%%%%%%%%%%%%%%%%%%%%%%%
% Informações do Trabalho
%%%%%%%%%%%%%%%%%%%%%%%%%%%%%%%%%%%%%%%%
\orientador[a]{\prof[a] \dr[a] Letícia Lopes Leite}{CIC/UnB}%
\coorientador{\prof \dr Raphael Moura Cardoso}{PPB/UnB}
\coordenador{\prof \dr Rodrigo Bonifácio de Almeida}{CIC/UnB}%
\diamesano{08}{dezembro}{2017}%

\membrobanca{\prof[a] \dr[a] Maria de Fátima Ramos Brandão}{CIC/UnB}%
\membrobanca{\prof[a] \dr[a] Letícia Lopes Leite}{CIC/UnB}%
\membrobanca{\prof \dr Edison Ishikawa}{CIC/UnB}%

\autor{Ana Carolina Lopes de}{Jesus}%
\coautor{Géssica Neves Sodré da}{Silva}%

\titulo{As Aventuras de Ada e Turing: Apoiando o Desenvolvimento do Pensamento Computacional por meio de um Jogo}%

\palavraschave{Pensamento Computacional, Ensino Fundamental, Jogos Digitais}%
\keywords{Computational Thinking, Elementary School, Digital Games}%

\newcommand{\unbcic}{\texttt{UnB-CIC}}%

%%%%%%%%%%%%%%%%%%%%%%%%%%%%%%%%%%%%%%%%
% Texto
%%%%%%%%%%%%%%%%%%%%%%%%%%%%%%%%%%%%%%%%
\begin{document}%
	
    \capitulo{1_Introducao}{Introdução\label{cap:intro}}%
    \capitulo{2_Tecnologias_na_Educacao}{Tecnologias na Educação\label{cap:tecnologia}}%
	\capitulo{3_Pensamento_Computacional}{O que é o Pensamento Computacional?\label{cap:pensamento}}%
    \capitulo{4_Jogos_Digitais}{Jogos Digitais\label{cap:jogos_digitais}}%
    \capitulo{5_O_Jogo}{As Aventuras de Ada e Turing\label{cap:jogo}}%
    \capitulo{6_Avaliacao_e_Testes}{Avaliação e Testes\label{cap:testes}}%
    \capitulo{7_Conclusao}{Conclusão\label{cap:conclusao}}%
    
	\apendice{Apendice_Design}{Documento de Design\label{ap:design}}%
	\apendice{Apendice_Termo}{Termo de Consentimento Livre e Esclarecido\label{ap:termo}}%
    \apendice{Apendice_Roteiro_Teste}{Roteiro de Teste\label{ap:roteiro}}%
    \apendice{Apendice_Certificado}{Certificado\label{ap:certificado}}%
    
\end{document}%