À nossa orientadora Letícia Lopes, pelo apoio indispensável, carinho, pelas suas correções e incentivo. É uma honra e grande felicidade pra nós sermos orientadas por uma mulher em uma área como a nossa.

Ao nosso coorientador Raphael Moura Cardoso, pela predisposição em nos auxiliar, pelas sugestões bem colocadas e apoio mesmo quando trazia assuntos fora do assunto da matéria.

À Marcela Paiva e ao Matheus Rinco pelo auxílio em momentos decisivos. Ao colega Dejaime Neto e ao Programa Infanto Juvenil (PIJ) pelo apoio durante a fase de testes.

A Universidade de Brasília, pelas oportunidades oferecidas.

A todos que direta ou indiretamente fizeram parte da nossa formação, o nosso muito obrigada.