A Ciência da Computação é utilizada de forma transversal em todas as áreas do conhecimento. Seu caráter interdisciplinar auxilia na evolução das mais diversas atividades humanas, como a construção de edifícios, a análise de DNA e a criação de medicamentos.  A transversalidade da Ciência da Computação também se reflete quando vemos tanto cientistas da computação sendo requisitados para desenvolver soluções em inúmeras áreas, como por exemplo a biologia, a medicina e o \textit{marketing}, quanto profissionais de outras modalidades buscando recursos computacionais que sejam capazes de auxiliar em suas respectivas profissões.

Desta maneira, evidencia-se uma forte influência da Ciência da Computação em outras áreas. Em 2008, Blikstein~\cite{blikstein_o_2008} destacava a época de transição que vivíamos, apontando como exemplo cientistas que passavam a maior parte do tempo construindo modelos computacionais ao invés de perpetuarem o clássico estereótipo do cientista do século XX, que passava horas em um laboratório com tubos de ensaio, e os engenheiros, que também utilizam modelos computacionais além de papel e lápis ao projetarem linhas de produção. Além disso, o autor também afirma que precisamos de mais pessoas que não sejam apenas usuários de tecnologia, mas também criadores de soluções.

Com a computação sendo aplicada em nosso dia a dia tanto na vida pessoal quanto no trabalho, é essencial que 
\begin{quote}
[...] nós sejamos capazes de entender como, quando, e onde computadores e outras ferramentas digitais podem nos ajudar a resolver problemas, além de precisarmos também saber como nos comunicar com outros indivíduos que possam nos ajudar com soluções computacionais.~\cite[~p.23, tradução das autoras]{barr_computational_2011}
\end{quote}

Neste contexto, situa-se o Pensamento Computacional, termo que vem ganhando destaque pelo mundo desde a década passada e que gerou diversas discussões quanto à sua inclusão no currículo escolar básico de forma direta ou transversal às demais disciplinas, com o objetivo de preparar indivíduos capazes de analisar problemas e projetar suas soluções, sejam computacionais ou não. Para melhor entendermos o cenário que envolve o tema, nos aprofundaremos em algumas das suas definições (seção~\ref{sec:definicao_pc}), a inclusão do Pensamento Computacional da Educação Básica (seção~\ref{sec:pc_educacao_basica}) e as habilidades que o compõem (seção~\ref{sec:habilidades}).

%%%%%%%%%%%%%%%%%%%%%%%%%%%%%%%%%%%%%%%%%%%%%%%%%%%%%%%%%%%%%%%%%%%%%%%%%%%%%%%
%%%%%%%%%%%%%%%%%%%%%%%%%%%%%%%%%%%%%%%%%%%%%%%%%%%%%%%%%%%%%%%%%%%%%%%%%%%%%%%
%%%%%%%%%%%%%%%%%%%%%%%%%%%%%%%%%%%%%%%%%%%%%%%%%%%%%%%%%%%%%%%%%%%%%%%%%%%%%%%
\section{Múltiplas Definições para Pensamento Computacional} \label{sec:definicao_pc}

Antes mesmo de haver uma discussão acerca do tema “Pensamento Computacional”, alguns autores já defendiam o aprendizado da programação. Em 1962, Alan Perlis~\cite{guzdial_paving_2008} destacou  a programação como um processo de exploração que dizia respeito a todos, além de ser um passo em direção à compreensão da “teoria da computação”, que levaria os estudantes a redefinirem seu entendimento acerca de uma variedade de tópicos. No contexto da Educação Básica, Papert~\cite{papert1994children} deu popularidade à ideia do uso de computação e foi o pioneiro no desenvolvimento do pensamento procedural através da linguagem LOGO.

No entanto, o termo Pensamento Computacional em si passou a ser discutido a partir de 2006, no artigo \textit{Computational Thinking}, de Jeannette Wing~\cite{wing_computational_2006}. Nele a autora afirma que o “Pensamento Computacional é construído com base no poder e nos limites dos processos de computação, sejam eles executados por humanos ou por máquinas”. No mesmo artigo, Wing fornece uma série de características delimitantes do que é ou não é considerado Pensamento Computacional: 

\begin{itemize}
\item conceituar, e não programar: Pensamento Computacional é mais do que saber programar e requer múltiplos níveis de abstração;
\item fundamental, não mecânico: uma habilidade não mecânica e que deve ser adquirida por todos;
\item forma como os humanos, e não as máquinas, pensam:  uma forma com a qual os seres humanos resolvem problemas, não uma tentativa de ensinar humanos a pensarem como máquinas;
\item complementa e combina a matemática e engenharia: Pensamento Computacional vale-se tanto da matemática quanto da engenharia;
\item ideias, não artefatos: computação não afeta a sociedade apenas por meio de softwares e hardwares, mas também através dos conceitos computacionais que são utilizados no cotidiano das pessoas;
\item para todos, em todos os lugares: Pensamento Computacional se tornará uma parte integral de todos as realizações humanas.
\end{itemize}

Apesar da disseminação do Pensamento Computacional, não houve uma definição concreta do termo, sendo a maior contribuição de Wing~\cite{wing_computational_2006} a respeito do tema a distinção entre Pensamento Computacional, programação e outros tipos de pensamentos analíticos, além de destacar a importância do mesmo para qualquer pessoa em todos os campos de atuação. Em 2008, Wing~\cite{wing_computational_2008} volta a fazer contribuições acerca do assunto, prenunciando o Pensamento Computacional como um instrumento para a descoberta e inovação em várias áreas e como parte integral da educação infantil e, consequentemente, do ensino do Pensamento Computacional para crianças.

Influenciados por Wing, diversos autores deram suas próprias definições de Pensamento Computacional. Dentre eles, Yadav, Hong e Stephenson~\cite{yadav_computational_2016} afirmam que o Pensamento Computacional envolve: dividir problemas complexos em subproblemas mais simples (decomposição de problemas) por meio de uma sequência de passos (algoritmos), revisar como a solução pode ser aplicada em problemas similares (abstração) e, por fim, determinar se um computador pode ajudar na resolução do problema.

Paulo Blikstein~\cite{blikstein_o_2008}, por sua vez, destaca que o Pensamento Computacional não é saber utilizar o computador em tarefas do dia a dia, como enviar e-mails e navegar na  internet,  mas sim usufruir do computador como “um instrumento de aumento do poder cognitivo e operacional humano” para aumentar a criatividade e a produtividade. Além disso, o autor também afirma que “pensar computacionalmente” envolve duas etapas: primeiro, é necessário identificar as tarefas cognitivas que podem ser otimizadas através do uso de um computador. A segunda etapa envolveria saber programar um computador para executar essas tarefas.

Embora existam várias similaridades entre as definições apresentadas, é possível verificar que Blikstein~\cite{blikstein_o_2008} considera a programação uma etapa fundamental do Pensamento Computacional, enquanto Wing~\cite{wing_computational_2006} defende que o Pensamento Computacional está além da programação e exige um poder de abstração mais elevado. Sendo este um ponto de divergência entre diversos autores.

Barr, V. e Stephenson ~\cite{barr_bringing_2011} estão entre os autores que evidenciam a função da programação no Pensamento Computacional como ferramenta que pode ser utilizada para implementar soluções. Em \textit{Bringing Computational Thinking to K-12}, os autores argumentam que:

\begin{quote}
O Pensamento Computacional é uma abordagem de resolução de problemas que podem ser implementados por um computador. Estudantes tornam-se não apenas usuários, mas construtores de ferramentas. Eles utilizam conceitos como abstração, recursividade e iteração para processar e analisar dados e criar artefatos tanto reais quanto virtuais. ~\cite[~p.51, tradução das autoras]{barr_bringing_2011}
\end{quote}

Apesar da falta de uma definição única para o Pensamento Computacional, vale ressaltar que mesmo fazendo parte dos seus conceitos centrais, como será apresentado na seção \ref{sec:habilidades}, a Ciência da Computação e o Pensamento Computacional são conceitos distintos. A \acrshort{ACM} em \textit{Model Curriculum for K-12 Computer Science}~\cite{tucker_model_2003} traz uma definição específica para educadores do ensino básico: “Ciência da computação é o estudo de processos de computadores e algoritmos, incluindo seus princípios, hardware e desenho de software, aplicações e impacto na sociedade”~\cite[~p.6, tradução das autoras]{tucker_model_2003}. 

Por outro lado, embora compartilhe elementos da matemática e da engenharia, o Pensamento Computacional diverge de outras áreas do conhecimento. Lee et al~\cite{lee_computational_2011} admitem que embora o Pensamento Computacional compartilhe de elementos de outros tipos de pensamento como o algorítmico e matemático, ele os complementa de forma única. 

Neste trabalho, será utilizada a definição formulada por Yadav, Hong e Stephenson~\cite{yadav_computational_2016}, uma vez que está de acordo com Barr, V. e Stephenson~\cite{barr_bringing_2011} quanto à implantação do Pensamento Computacional na Educação Básica, uma das proposições deste trabalho. Ademais, é uma definição mais recente, clara e concisa que a lista de características propostas por Wing~\cite{wing_computational_2006} e não trata a programação como um aspecto essencial do Pensamento Computacional, como Blikstein~\cite{blikstein_o_2008}.

%%%%%%%%%%%%%%%%%%%%%%%%%%%%%%%%%%%%%%%%%%%%%%%%%%%%%%%%%%%%%%%%%%%%%%%%%%%%%%%
%%%%%%%%%%%%%%%%%%%%%%%%%%%%%%%%%%%%%%%%%%%%%%%%%%%%%%%%%%%%%%%%%%%%%%%%%%%%%%%
%%%%%%%%%%%%%%%%%%%%%%%%%%%%%%%%%%%%%%%%%%%%%%%%%%%%%%%%%%%%%%%%%%%%%%%%%%%%%%%
\section{Pensamento Computacional na Educação Básica}\label{sec:pc_educacao_basica}

Conforme apresentado anteriormente, dada a importância da computação e dos seus desdobramentos como ferramenta indispensável ao avanço da sociedade, é necessário preparar indivíduos capazes de criar soluções. Wing~\cite{wing_computational_2006} foi a primeira autora a sugerir que é preciso que as bases do Pensamento Computacional tornem-se comuns a todas as áreas, assim como outros tipos de pensamento, como o matemático e o científico, que são desenvolvidos durante toda a vida acadêmica.

Outrossim, diversos autores vêm defendendo a introdução do Pensamento Computacional como uma disciplina obrigatória da Educação Básica. Barr, V. e Stephenson~\cite{barr_bringing_2011} defendem que não é mais suficiente esperar que os estudantes cheguem até o ensino superior para serem expostos aos conceitos do Pensamento Computacional.  

Yadav, Hong e Stephenson~\cite{yadav_computational_2016} argumentam que uma vez que o Pensamento Computacional é focado na resolução de problemas e promove o engajamento dos estudantes no planejamento de processos que podem ser automatizados, é essencial que educadores do K-12\footnote{K-12 é o termo equivalente à Educação Básica nos Estados Unidos e abrange 12 anos. Mais informaçõe sobre K-12 podem ser encontradas no site: http://dmmcihs.edu.ph/7/senior-high-school/k12-prog-definition} e administradores explorem formas de incorporar as suas ideias ao currículo, pois conectar os conceitos do Pensamento Computacional às disciplinas que os professores já ensinam seria a melhor abordagem de ensino para satisfazer as demandas curriculares atuais. Similarmente, Barr, D., Harrisson e Conery~\cite{barr_computational_2011} defendem que embora os estudantes já aprendam alguns elementos que fazem parte do conjunto de habilidades do Pensamento Computacional, é necessário garantir que eles tenham a oportunidade de aprender o conjunto completo de habilidades, para que possam usufruir do seu poder conjunto.
 
Além da comunidade acadêmica, empresas de tecnologia também têm demonstrado interesse em promover a disseminação do Pensamento Computacional, como a \textit{Microsoft}, que em 2007 criou o \textit{Center for Computational Thinking} em parceria com a Universidade de Carnegie e Mellon\footnote{https://www.cs.cmu.edu/~CompThink/}, e o \textit{Google}, que oferece cursos online para educadores e administradores que têm interesse em se informar sobre o Pensamento Computacional e trazê-lo para as escolas\footnote{https://edu.google.com/resources/programs/exploring-computational-thinking/}.

No contexto escolar, em 2014, Judith Gal-Ezer e Chris Stephenson~\cite{gal-ezer_tale_2014} contaram a história de sucesso e os desafios enfrentados no desenvolvimento do currículo/padrões do ensino de Ciência da Computação em Israel e nos Estados Unidos, destacando o Pensamento Computacional como uma habilidade da Ciência da Computação que os estudantes devem desenvolver e, para tanto, recomendam a sua inclusão como conteúdo curricular. 

Além de Israel e dos Estados Unidos, outros países também vêm se empenhando na expansão do Pensamento Computacional, como a Nova Zelândia, que irá incorporá-lo ao seu currículo como disciplina essencial a todas as crianças nos 10 primeiros anos escolares~\cite{parsons_2016} e o Reino Unido, que sofreu uma alteração curricular em 2014, na qual propôs a inserção do Pensamento Computacional no ensino secundário~\cite{cas_2014}.  

No Brasil, assim como nos Estados Unidos, também vêm ocorrendo esforços conduzidos pela academia com o intuito de disseminar o Pensamento Computacional. Os projetos normalmente utilizam materiais baratos como papelão e garrafas em atividades lúdicas no intuito de promover habilidades do Pensamento Computacional~\cite{pinho_proposta_2016}. Embora o ideal fosse utilizar os recursos tecnológicos em tais atividades a fim de que os alunos vivenciassem as tecnologias que irão encontrar fora da sala de aula, é preciso observar que a realidade brasileira torna inviável que isso ocorra em todas as escolas. Em pesquisa, a Fundação Vitor Civita~\cite{fundacao_victor_civita_o_2009} aponta que “questões de infraestrutura, como número reduzido de computadores e falta de um laboratório de informática, são vistas como o principal problema no uso pedagógico”, além do despreparo dos professores para lidar com a tecnologia. Ademais, Yadav, Hong e Stephenson~\cite{yadav_computational_2016} afirmam que: 

\begin{quote}
Embora seja valioso para os estudantes poderem aprender num contexto de currículo de Ciência da Computação e ambientes de programação, as limitações de uma sala de aula dentro do K-12 inviabilizam o acesso a todas as escolas a cursos voltados somente à computação. No entanto, as ideias do Pensamento Computacional são interdisciplinares e podem ser inseridas ao Ensino Fundamental e Secundário.~\cite[~p.566, tradução das autoras]{yadav_computational_2016}
\end{quote}

Dentre os esforços brasileiros que têm como objetivo disseminar o Pensamento Computacional, pode-se citar os pesquisadores André Raabe e André Santana~\cite{santana_atividades_2016}~\cite{santana_lite_2016}, que vêm participando, juntamente com outros colaboradores da \acrshort{UNIVALI}, de projetos que difundem o Pensamento Computacional por meio de atividades \textit{makers} \footnote{O movimento \textit{maker} tem relação direta com o termo “faça você mesmo” e tem como objetivo desenvolver o potencial criativo do indivíduo através de atividades onde ele mesmo põe a “mão na massa” criando desde artesanatos a soluções tecnológicas~\cite{tanji_makers:_2017}.} no contexto de ambientes construcionistas, os chamados \textit{fab labs} ou laboratórios de fabricação, que reúnem um conjunto de tecnologias digitais e físicas com o propósito de desenvolver habilidades inventivas e criativas colaborativamente.

Outros exemplos de trabalhos desenvolvidos com o propósito de propagar o Pensamento Computacional são o estudo realizado com crianças do quarto ano na assimilação de habilidades inerentes ao Pensamento Computacional através de uma atividade relacionada a números binários~\cite{campos_organizacao_2014} e, similarmente, a tentativa realizada por Hinterholz e Cruz~\cite{hinterholz_desenvolvimento_2015} por meio do ensino de conceitos de bancos de dados. 

Ainda no contexto da Educação Básica, deve-se notar que o desenvolvimento de um currículo que inclua o Pensamento Computacional em sua base, ou mesmo o uso de forma independente em instituições que não o tenham em seu currículo formal, exige a distinção das habilidades trabalhadas por ele. A seção \ref{sec:habilidades} versa sobre o tema e traz algumas definições.

%%%%%%%%%%%%%%%%%%%%%%%%%%%%%%%%%%%%%%%%%%%%%%%%%%%%%%%%%%%%%%%%%%%%%%%%%%%%%%%
%%%%%%%%%%%%%%%%%%%%%%%%%%%%%%%%%%%%%%%%%%%%%%%%%%%%%%%%%%%%%%%%%%%%%%%%%%%%%%%
%%%%%%%%%%%%%%%%%%%%%%%%%%%%%%%%%%%%%%%%%%%%%%%%%%%%%%%%%%%%%%%%%%%%%%%%%%%%%%%
\section{Habilidades do Pensamento Computacional}\label{sec:habilidades}

Com a popularização do Pensamento Computacional e o reconhecimento da comunidade acadêmica quanto à sua importância, existem diversos esforços com o objetivo de conceber uma lista que abarque as suas habilidades básicas. 

Tendo em vista a criação de uma estrutura e vocabulário de apoio que tornasse os conceitos do Pensamento Computacional acessíveis aos educadores, a \acrshort{CSTA}/\acrshort{ISTE}~\cite{iste/csta_operational_2011} propôs uma definição operacional do Pensamento Computacional. Essa definição o apresenta como um processo de resolução de problemas que inclui as seguintes características:
\begin{itemize}
\item elaborar problemas de forma que seja possível utilizar um computador e outras ferramentas para resolvê-los;
\item organizar e analisar dados logicamente;
\item representar dados por meio de abstrações, como modelos e simulações;
\item automatizar soluções por meio de pensamento algorítmico;
\item identificar, analisar e implementar possíveis soluções com o intuito de atingir a combinação mais eficiente e efetiva de passos e recursos;
\item generalizar e transferir esse processo de resolução de problemas para outros problemas.
\end{itemize}

Formuladas durante um \textit{workshop} promovido também pela \acrshort{CSTA}, Barr, V. e Stephenson~\cite{barr_bringing_2011} apresentaram listas resultantes de discussões dos conceitos centrais do Pensamento Computacional no contexto de habilidades, disposições e pré-disposições, e cultura de sala de aula. O resultado das discussões acerca das habilidades incluem:
\begin{itemize}
\item projetar soluções para problemas utilizando abstração, automação, criação de algoritmos, coleta de dados e análise;
\item implementar modelos;
\item realizar testes e depuração;
\item modelar, executar simulações e analisar sistemas;
\item refletir sobre prática e comunicação;
\item usar o vocabulário;
\item reconhecer abstrações e ser capaz de trabalhar com diferentes níveis de abstração;
\item inovar, explorar e exercer a criatividade interdisciplinarmente;
\item resolver problemas em grupo;
\item empregar diferentes estratégias de aprendizagem.
\end{itemize}

Analisando esses conjuntos de habilidades, verifica-se que a segunda lista inclui a realização de testes e depuração, comunicação, uso de vocabulário, trabalho em grupo e o emprego de estratégias de aprendizagem, enquanto a outra exclui esses itens. Apesar disso, as outras habilidades são muito similares, embora sejam expressas de formas diferentes.

No âmbito deste trabalho, o conjunto de habilidades apresentados por Barr, V. e Stephenson~\cite{barr_bringing_2011} serão utilizados no desenvolvimento de um jogo digital direcionado a crianças cursando o Ensino Fundamental I (a partir dos 6 anos). O uso de habilidades discutidas em um contexto de profissionais da área de educação justifica-se pela importância de uma base teórica adequada que possa guiar o desenvolvimento de um jogo educativo. Ademais, a escolha da lista que possui um número maior de habilidades se deve ao fato de que ela inclui a comunicação e o trabalho em grupo, habilidades muito valorizadas no mercado de trabalho atual e que poderão incentivar o engajamento dos alunos durante as atividades do jogo. 

Apesar dos desafios que a inclusão do Pensamento Computacional no currículo básico representa, principalmente no contexto brasileiro, devido à falta de infraestrutura das escolas e do despreparo dos professores para introduzi-lo ao contexto escolar, é possível utilizar recursos alternativos, como os jogos digitais, para desenvolver as habilidades que o compõem. Embora não substituam a necessidade de um novo currículo que inclua o Pensamento Computacional, os jogos podem ser aliados poderosos na sua disseminação. 
