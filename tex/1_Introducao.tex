Desde o Altair 8800, anunciado em 1975 e considerado por muitos o primeiro computador pessoal devido ao seu preço~\cite{computer_campbell_2013}, muita coisa mudou. Nas últimas décadas, o mercado tecnológico observou a introdução de \textit{notebooks}, \textit{tablets} e \textit{smartphones}, além do crescente acesso dessas tecnologias pelas mais diversas camadas da sociedade.

Apesar da inegável e crescente disseminação dos computadores nos últimos anos, um estudo realizado entre 2011 e 2015 pela \textit{\acrfull{OECD}}~\cite{oecd_shills_2016}, em 33 países, com indivíduos com idades entre 16 e 65 anos de idade, mostrou que cerca de 26\% da população adulta não sabe utilizar computadores, e que apenas 5\% é capaz de conceber soluções consideradas mais complexas e que exigem o uso tanto de tecnologias computacionais genéricas quanto mais específicas.

Isso mostra que a maioria dos usuários de computadores os utilizam de forma superficial e poucos sabem aplicá-los em tarefas mais complexas, o que pode se tornar um problema no futuro, devido ao rápido crescimento de ocupações na área de tecnologia. Essa expansão pode ser evidenciada por um relatório da \textit{Burning Glass}~\cite{beyond_burning_2016}, uma empresa especializada em estudos analíticos voltados ao mercado de trabalho, que afirma que empregos que exigem conhecimentos em programação estão crescendo 12\% mais rápido que a média geral do mercado para os demais tipos de ocupações.  

Tendo esses dados em vista, pode-se supor que, no futuro, os indivíduos que souberem utilizar computadores somente de maneira básica no dia a dia, limitando-se a atividades como leitura de \textit{e-mails} e utilização de redes sociais, terão poucas oportunidades de trabalho num mercado que vem se tornando cada vez mais competitivo e, consequentemente, ficarão em séria desvantagem com relação aos que conhecem e utilizam as tecnologias tanto em seu cotidiano como para a evolução das suas áreas de atuação profissional. 

No contexto dos avanços tecnológicos constantes em que vivemos, surgiu o tema Pensamento Computacional, termo que ganhou importância ao ser apresentado por Wing~\cite{wing_computational_2006, wing_computational_2008, wing_computational_2011} em seus trabalhos e que representa um esforço em busca do desenvolvimento de suas habilidades e competências em todos os indivíduos, independente da área de atuação.

Dados os benefícios oferecidos pelos conhecimentos computacionais à inserção social de um indivíduo, tais como vantagem competitiva no mercado de trabalho e aquisição de habilidade de lidar com problemas complexos - entendendo o que é o problema e desenvolvendo soluções para ele, de maneira computacional ou não - que podem ser utilizadas nos mais diversos campos de atuação, o foco deste trabalho será no Pensamento Computacional.

Além disso, este trabalho tratará de como promover o desenvolvimento do Pensamento Computacional nas séries do Ensino Fundamental I no Brasil e orienta-se através das seguintes questões: 

\begin{itemize}
\item O que é Pensamento Computacional?
\item Por que o Pensamento Computacional deve ser desenvolvido desde o Ensino Fundamental I?
\item Como o Pensamento Computacional pode ser desenvolvido no Ensino Fundamental I?
\end{itemize}

Em resposta à terceira pergunta - como o Pensamento Computacional pode ser desenvolvido no Ensino Fundamental I - é levantada a seguinte hipótese: por meio de uma abordagem de jogos educacionais, é possível desenvolver habilidades do pensamento educacional no Ensino Fundamental I de forma lúdica e sem a necessidade de alteração dos currículos nacionais.

Este trabalho tem como objetivo geral \textbf{apoiar o desenvolvimento de habilidades do Pensamento Computacional em alunos do Ensino Fundamental I}. E, como objetivos específicos:

\begin{itemize}
	\item apropriação da temática do Pensamento Computacional;
	\item levantamento de usos dos jogos digitais no desenvolvimento do Pensamento Computacional;
	\item proposição de protótipo de jogo como apoio ao desenvolvimento de habilidades do Pensamento Computacional;
	\item realização da avaliação do jogo desenvolvido, visando identificar possíveis manutenções na ferramenta;
	\item realização de testes com especialistas na área e com possíveis usuários do jogo.
\end{itemize}

A metodologia utilizada envolve pesquisa exploratória para estudo das abordagens relacionadas ao desenvolvimento de habilidades do Pensamento Computacional na educação básica e desenvolvimento de um jogo digital que desenvolva um subconjunto dessas habilidades. Além disso, foi realizada a avaliação dos mecanismos de um jogo através do \textit{framework Octalysis}, aspectos de jogabilidade e da realização de teste de usabilidade com o público alvo.

Espera-se, através deste trabalho, despertar a curiosidade acerca do Pensamento Computacional e ressaltar a importância do seu desenvolvimento, desde as séries do Ensino Fundamental. Ademais, apresentar o protótipo do jogo desenvolvido intitulado “\textbf{As Aventuras de Ada e Turing}”.

Este trabalho está dividido em sete capítulos. O capítulo~\ref{cap:tecnologia} apresenta a tecnologia como ferramenta para a Educação. O capítulo~\ref{cap:pensamento} aborda o Pensamento Computacional, tema base deste trabalho. O capítulo~\ref{cap:jogos_digitais} apresenta os Jogos Digitais como ferramenta para se trabalhar diversos temas no âmbito da Educação. O capítulo~\ref{cap:jogo} apresenta o protótipo do jogo “\textbf{As Aventuras de Ada e Turing}”, enquanto o capítulo~\ref{cap:testes} aborda os avaliações e os testes realizados. Finalmente, no capítulo~\ref{cap:conclusao} são apresentadas as conclusões e os trabalhos futuros.