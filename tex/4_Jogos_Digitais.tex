Os jogos têm sido um passatempo humano há séculos. Desde os simples jogos de dados e cartas aos complexos jogos de tabuleiro, eles têm sido uma forma de entretenimento poderosa capaz de aliviar o estresse da vida cotidiana~\cite{netzley_how_2015}. Em \textit{Homo Ludens}, Huizinga~\cite{huizinga_homo_1980} discute a importância dos jogos como um fenômeno cultural, afirmando que as atividades humanas são permeadas pelos jogos desde o princípio e que estes antecedem a própria cultura. 

Com o advento dos computadores, surgiram também os jogos digitais, que alteram  a maneira como as pessoas jogam a cada evolução tecnológica. Os primeiros jogos que surgiram foram desenvolvidos unicamente por programadores e a tecnologia da época não permitia muitas opções para a interface dos jogos~\cite[~p. 16-50]{mendes_estilo_2013}, a exemplo do Pong de 1972 (Figura~\ref{fig:pong}). Este simula um jogo de tênis de mesa, e possui apenas duas barras verticais que se movem para cima e para baixo e uma esfera que se move de um lado para o outro da tela com a colisão contra a barra, simulando as raquetes e uma bola, respectivamente. Com gráficos e mobilidade tão limitados, as possibilidades do jogo eram restringidas e dificilmente aplicadas em outro contexto fora do entretenimento.

\figura[H]{pong}{Primeira geração do jogo Pong com Bushnell, um dos criadores, ao centro}{fig:pong}{width=0.4\textwidth}

As evoluções tecnológicas viabilizaram a migração de jogos digitais de gabinetes da altura de pessoas, para dispositivos móveis, como óculos de realidade virtual e celulares, proporcionando não só portabilidade, mas aumento da capacidade de processamento e, consequentemente, jogos mais poderosos. As possibilidades são diversas, e estudos são feitos inclusive na área da saúde, onde jogos de realidade virtual estão sendo testados, entre outras funções, para a reabilitação postural de pacientes~\cite{lima_reabilitacao_2017}.

Segundo a Pesquisa Game Brasil\footnote{\url{https://www.pesquisagamebrasil.com.br/}}~\cite{sioux_pesquisa_2017}, em 2017, 77.9\% dos jogadores costumam utilizar \textit{smartphones} para jogar, enquanto 49.0\% utilizam \textit{consoles} e 66.4\% computador. Acrescentam ainda, no quesito \textit{smartphones}, que 71.0\% do total de entrevistados jogam em casa, 60.7\% no trânsito e 45.4\% no trabalho, o que aponta que a mobilidade propiciada pelos dispositivos móveis, em especial os \textit{smartphones},  expandiu os locais onde é possível jogar.

Em \textit{How do Video Games Affect Society}, Netzley~\cite{netzley_how_2015} analisa os impactos dos \textit{videogames} sob diferentes pontos de vista, fornecendo uma discussão ampla sobre os benefícios e malefícios que eles podem acarretar. Dentre os malefícios, a autora cita distúrbios do sono, problemas de visão, dores de cabeça e até o desencadeamento de comportamentos violentos. No entanto, Netzley deixa claro que os efeitos negativos ou positivos no comportamento de um indivíduo dependem fortemente do tipo de jogo, citando um estudo realizado em 2013 por Adachi e Willoughby~\cite{adachi_more_2013}, no qual foi relatado que adolescentes que jogaram \textit{videogames} de estratégia, particularmente os \textit{role-playing games}, obtiveram melhoras em suas avaliações acadêmicas e habilidades relacionadas à resolução de problemas.

Com a inclusão cada vez maior dos jogos em contextos diferentes do puro entretenimento, surge o termo Jogos Sérios (do inglês \textit{Serious Games}). Dentre as várias definições, o conceito em questão trata do uso de jogos além da diversão, a exemplo dos jogos educacionais - jogos sérios aplicados para a educação - como para o ensino de matemática, desenvolvimento de habilidades cognitivas, computação, treinamento cirúrgico, entre outros~\cite{backlund_educational_2013}.

Na educação, os jogos digitais tendem a aparecer como ferramenta motivadora, buscando engajar os alunos em assuntos pelos quais não necessariamente se interessariam~\cite{paula_jogos_2016}. Porém, alguns deles se limitam a testar os conhecimentos do aprendiz e outros são estrangeiros, adaptados a um sistema educacional diferente. Para serem utilizados de forma mais efetiva em sala de aula, os jogos precisam ser pedagogicamente estruturados, promovendo situações interessantes e desafiadoras para a resolução de problemas, além de permitir auto-avaliação quanto aos seus desempenhos~\cite{moratori_por_2003}. Entretanto, estruturar pedagogicamente um jogo não garante a sua efetividade, para tanto algumas iniciativas podem contribuir para o alcance deste objetivo, tais como: formação do docente e uma proposta pedagógica que utilize o jogo como apoio ao processo de ensino e aprendizagem~\cite{araujo_construcao_2016,lopes_videojogos_2013}.

A ciência e a tecnologia estão por toda a parte, publicações da Cetic.br\footnote{\url{http://cetic.br/publicacoes/indice/pesquisas/}} evidenciam  o uso das \acrshort{TIC} em empresas, estabelecimentos de saúde, domicílios, governo e escolas.  Assim, a adoção de tecnologias digitais, a exemplo dos jogos digitais, busca diminuir as discrepâncias entre o que é visto dentro da sala de aula e a realidade vivida pela sociedade. Paula e Valente~\cite{paula_jogos_2016} destacam os jogos digitais como uma abordagem para a modificação do processo educacional, favorecendo a aprendizagem significativa, de forma que o potencial da utilização de jogos digitais na educação se encontre nas ações realizadas pelo jogador para superar o desafio, não necessariamente no “conteúdo” apresentado pelo jogo.

No Brasil, alguns esforços estão sendo realizados para a utilização de jogos digitais na educação. No contexto de informática na educação, Medeiros, Silva e Aranha~\cite{medeiros_ensino_2013} destacam os jogos digitais como ferramenta no ensino de programação em artigos publicados entre os anos de 2008 e 2012. Zanetti, Borges e Ricarti~\cite{zanetti_pensamento_2016} tratam do Pensamento Computacional como tema no ensino de linguagens de programação em eventos brasileiros de 2012 até 2015, destacando alguns artigos que utilizam jogos digitais como ferramenta de ensino e aprendizagem. Seguindo esta linha, neste capítulo aprofundaremos no uso de jogos digitais na educação.

%%%%%%%%%%%%%%%%%%%%%%%%%%%%%%%%%%%%%%%%%%%%%%%%%%%%%%%%%%%%%%%%%%%%%%%%%%%%%%%
%%%%%%%%%%%%%%%%%%%%%%%%%%%%%%%%%%%%%%%%%%%%%%%%%%%%%%%%%%%%%%%%%%%%%%%%%%%%%%%
%%%%%%%%%%%%%%%%%%%%%%%%%%%%%%%%%%%%%%%%%%%%%%%%%%%%%%%%%%%%%%%%%%%%%%%%%%%%%%%

\section{Jogos Digitais na Educação} \label{sec:jogos_digiais_utilizacao}

O advento e a popularização dos jogos digitais em diversas faixas etárias, e principalmente entre os mais jovens, proporcionou sua utilização como ferramenta potencializadora de novas e enriquecedoras oportunidades de aprendizagem. Hoje, temos exemplos de algumas abordagens para trazer os jogos para a sala de aula, principalmente como instrumento que engaje o aluno e aumente sua motivação para aprender.

Analisando as abordagens utilizadas para o Ensino Fundamental, que envolvem crianças entre 6 e 14 anos de idade, observamos alguns jogos digitais comerciais (jogos não-educacionais) adaptados para uso educacional. Falcão e Barbosa~\cite{falcao_aperta_2015}, destacam três jogos com abordagens similares para o ensino de lógica de programação: \textit{Hour of Code}\footnote{hourofcode.com}, \textit{The Foos}\footnote{www.thefoos.com} e \textit{Lightbot}\footnote{lightbot.com}. Nesta pesquisa, realizada com crianças abordadas na rua, os autores avaliaram o jogo \textit{Lightbot} como uma ferramenta exploratória que envolve o Pensamento Computacional e, ainda que comercial, tem sido usado por professores e pessoas que estão programando pela primeira vez para o aprendizado de conceitos de programação como sequenciamento, procedimentos, \textit{loops}, etc. No contexto da sala de aula, Cechin et al.~\cite{cechin_adaptacao_2012} refletem sobre o resultado da aplicação do jogo \textit{Angry Birds} adaptado para o ensino de física e \textit{Puzzle Quest} para o resgate de valores éticos e morais, ambos visando aprendizes que se encontram no final do Ensino Fundamental.

Adequar jogos comerciais já existentes criando ambientes de aprendizagem mais atraentes para o ensino de assuntos que não despertam tanto interesse dos alunos é uma das formas utilizadas por alguns professores e defendidas por pesquisadores para a inclusão das \acrshort{TIC} no meio escolar. Alguns fatores influenciam na escolha dessa abordagem: redução do custo para a criação do jogo, o fato de ser conhecido pelo público em outro cenário (possibilitando a implantação de um jogo já aceito pelos usuários como atrativo e divertido)~\cite{cechin_adaptacao_2012}. Porém, a escolha deste método exige que o jogo seja estudado pelo educador de forma a integrá-lo ao cenário adequado tanto pedagogicamente quanto em relação ao nível acadêmico do educando.

Um jogo digital criado com propósito educacional, idealmente, deve observar condições básicas  de qualidade. Franciosi~\cite{franciosi_modelagem_1997}, aponta os seguintes requisitos didático-pedagógicos como integrantes no planejamento de um software educacional:

\begin{itemize}
\item objetivos bem definidos;
\item encadeamento lógico do conteúdo;
\item adequação do vocabulário;
\item possibilidade de formação de conceitos;
\item correção da palavra escrita (ortografia e gramática);
\item feedback apropriado;
\item clareza e concisão dos textos apresentados;
\item possibilidade de acesso direto a diferentes níveis do programa;
\item possibilidade do professor incluir/excluir/alterar conteúdos do sistema.
\end{itemize}

Ademais, também é necessário considerar os requisitos de qualidade técnica a seguir:
\begin{itemize}
\item execução rápida e sem erros;
\item resistência a respostas inadequadas;
\item interface amigável;
\item tempo suficiente de exibição das telas;
\item possibilidade de acesso a ajuda;
\item possibilidade de trabalho interativo;
\item possibilidade de controle do usuário sobre a seqüência de execução do sistema;
\item possibilidade de correção de respostas;
\item possibilidade de sair do sistema a qualquer momento;
\item uso de telas com diagramação segundo um modelo único de organização.
\end{itemize}

Cumprir todos os requisitos listados anteriormente não é tarefa fácil, porém, jogos digitais desenvolvidos com propósito educativo têm seus objetivos mais claros para a aplicação em sala de aula. A exemplo temos o jogo “Adivinhas sobre a Tabela Periódica”~\cite{ramos_adivinhas_2004}, disponibilizado \textit{online}, utilizado para a compreensão de alguns aspectos da Tabela Periódica em disciplinas de Química com foco no Ensino Médio e que foi criado juntamente com outros jogos para o ensino de química como parte da tese de mestrado na Universidade do Porto~\cite{ramos_utilizacao_2004}. Nele, com base em textos fornecidos, os alunos tentam adivinhar qual é o elemento químico em questão com uma aproximação mais real com seu uso aplicando em situações do cotidiano.

Ainda que desenvolvido em uma universidade portuguesa, o jogo Adivinhas sobre a Tabela Periódica foi utilizado no Brasil com o propósito de apresentar a tabela periódica para alunos do 3º ano do Ensino Médio, do Educandário Menino Jesus de Praga, no município de Esperança. Lima e Moita~\cite[~p. 142]{lima_tecnologia_2011} entendem o jogo digital como uma forma de “adequar e inovar a metodologia de ensino de química, através da utilização de recursos tecnológicos que promovem qualidade ao processo de ensino e aprendizagem”, reconhecendo a importância da inclusão da tecnologia no ensino como estratégia metodológica. Ao final da pesquisa, os autores relacionaram o uso dos jogos digitais à prática pedagógica, com o objetivo de atingir o público jovem, mostrando mais uma experiência do uso com sucesso de jogos digitais, desta vez no Ensino Médio.

Para além do uso como ferramenta, tanto em sala de aula quanto em outros contextos educativos, os fundamentos dos jogos também surgem como artefatos para a construção de novas formas de motivar estudantes e aprimorar suas habilidades. Neste sentido, o conceito de gamificação foi trazido para o âmbito educacional. Segundo Chou~\cite{chou_what_2017}, a gamificação consiste na utilização dos elementos divertidos e atrativos encontrados nos jogos, aplicando-os ao mundo real para atividades produtivas. Desta forma, ao invés do foco estar somente na eficiência do sistema, a gamificação é um processo de design que enfatiza o humano no sistema, o que Chou denomina como “Design Focado no Humano”\footnote{No texto original em inglês o autor utiliza o termo “\textit{Human-Focused Design}”}.  A partir dessa análise, Chou empregou esses elementos para criar o \textit{Octalysis} (Seção~\ref{ssec:octalysis}), um \textit{framework} que auxilia na identificação do quão gamificado um produto ou processo está.

Freitas et al.~\cite{freitas_gamificacao_2016} empregaram a estratégia gamificada no projeto, implementação e teste do jogo “FAC – o jogo: Batalha do conhecimento”, na disciplina \acrfull{FAC} da \acrfull{FGA}, da \acrfull{UnB}. \acrshort{FAC} é uma disciplina do quarto semestre ofertada aos cursos de graduação em Engenharia de Software e Engenharia Eletrônica. Observando os estudantes \textit{in loco} e os dados analisados ao final da realização da estratégia, os autores constataram que:

\begin{quote}
[...] (1) as novas gerações estão ambientadas com a dinâmica dos jogos, (2) os aspectos motivacionais da gamificação induzem os usuários a apenas jogarem, sem maiores comprometimentos e (3) a aprendizagem gamificada induz naturalmente o estudante a aprender com prazer.~\cite[~p.378]{freitas_gamificacao_2016}
\end{quote}

Em conjunto a esses pontos, destacam-se os aspectos positivos da estratégia gamificada em sala de aula resultando em um aumento na dedicação dos estudantes e na qualidade de suas participações no percurso desenvolvido na disciplina.

Autores como Zanetti, Borges e Ricarte~\cite{zanetti_pensamento_2016}, Lopes e Oliveira~\cite{lopes_videojogos_2013}, Freitas et al.~\cite{freitas_gamificacao_2016},  Alves, Correia, Talissa e Correia, Thais~\cite{correia_revisao_2016} e Pereira et al.~\cite{pereira_jogos_2016} verificaram em seus estudos aspectos motivacionais e fomentadores do desenvolvimento de competências através da utilização de jogos digitais em diferentes áreas do conhecimento (matemática e programação são os temas centrais de destaque). Encontramos também esforços na literatura~\cite{araujo_construcao_2016} para a formação de professores do ensino básico com o intuito de utilizarem as etapas iniciais do \textit{Game Design}\footnote{Shell~\cite{schell_art_2014}, fornece uma definição ampla para \textit{game design} como o ato de decidir o que o jogo deve ser.} na criação de jogos digitais que auxiliem o processo de ensino e aprendizagem. Ainda, existem pesquisas quanto aos fatores que podem motivar o usuário a utilizar um jogo e, nesse contexto, está o \textit{Octalysis} (Seção~\ref{ssec:octalysis}). Entretanto, ainda há um número pequeno de pesquisas acerca da sua influência em diferentes contextos e as características que mais se desenvolvem em indivíduos jogadores do que nos não-jogadores~\cite{correia_revisao_2016}.

\subsection{O \textit{Framework} Octalysis} \label{ssec:octalysis}

Segundo Ramos, Lorenset e Petri~\cite{ramos_jogos_2016}, as características e elementos presentes nos jogos, como  regras e restrições, narrativa, objetivos, interação, desafio, competição e conflito, resultados, recompensas e \textit{feedback} contribuem à aprendizagem. Tais aspectos podem ser avaliados através do \textit{framework Octalysis}, que identifica oito direcionadores principais da gamificação apontados por Yu-kai Chou~\cite{chou_octalysis_2015}, formando uma estrutura de octógono (\refFig{fig:gamification-octalysis}).

\figura[H]{gamification-octalysis}{Oito direcionadores posicionados no Octógono~\cite{chou_octalysis_2015}}{fig:gamification-octalysis}{width=0.8\textwidth}

No desenvolvimento do \textit{framework Octalysis}, Yu-kai Chou~\cite{chou_octalysis_2015} indica que, em relação à diversão em um jogo, existem oito direcionadores principais que motivam a realização de atividades. Esses direcionadores são a base do octógono que forma o  \textit{framework Octalysis}:

\begin{enumerate}
\item Significado Épico \& Chamado: o motivador é a crença de que se está fazendo algo maior ou que foi o “escolhido” para fazer algo;
\item Desenvolvimento \& Realização: o motivador é o desenvolvimento de habilidades e, eventualmente, superação de desafios;
\item Empoderamento da Criatividade \& \textit{Feedback}: o motivador é o envolvimento em um processo criativo onde os usuários devem descobrir coisas repetidamente e tentar diferentes combinações;
\item Propriedade \& Posse: o motivador é o sentimento de que se possui algo;
\item Influência Social \& Relacionamento: os motivadores são elementos sociais que direcionam as pessoas, incluindo mentoria, aceitação, respostas sociais, companhia, competição e inveja;
\item Escassez \& Impaciência: o motivador é o desejo por algo que não se pode ter;
\item Imprevisibilidade \& Curiosidade: o motivador é o desejo de descobrir o que vai acontecer em seguida;
\item Perda \& Evasão: o motivador é a busca por evitar que algo negativo aconteça.
\end{enumerate}

Segundo Yu-kai Chou~\cite{chou_octalysis_2015}, além dos oito direcionadores descritos, o octógono gerado pelo \textit{framework Octalysis} também pode ser dividido em lado direito e esquerdo (respectivamente, \textit{Right Brain Core Drives} e \textit{Left Brain Core Drives}), e gamificadores superiores e inferiores (\textit{White Hat Gamification} e \textit{Black Hat Gamification}, respectivamente).  Os direcionadores do lado direito estariam relacionados à criatividade, auto-expressão e aspectos sociais, e o lado esquerdo mais associado à lógica, cálculos e pertencimento. Do mesmo modo, os direcionadores superiores são considerados motivadores muito positivos, enquanto os inferiores são classificados como negativos.

Os direcionadores descritos são a base da \textit{Octalysis Tool}\footnote{Ferramenta criada por Ron Bentata e disponibilizada gratuitamente no site http://www.yukaichou.com/octalysis-tool/} e, a partir dela, é possível avaliar se um produto ou processo apresenta características gamificadas. Cada um dos direcionadores é pontuado com um valor entre 0 e 10, baseado em julgamento pessoal, dados e experiência de uso. Quando toda pontuação for adicionada, é gerada a pontuação final do \textit{Octalysis}. Essa avaliação é importante para apontar o que está faltando no jogo e o que poderia ser trabalhado para melhorar a experiência do jogador e alcançar o objetivo pretendido com o produto ou projeto.

Yu-kai Chou~\cite{chou_octalysis_2015} informa que o \textit{framework Octalysis} aponta que qualquer produto ou sistema atraente terá, ao menos, um dos direcionadores apresentados, de forma que os mecanismos de jogo identificados durante a avaliação devem ser dispostos próximos ao direcionador no octógono. Baseado em quão forte o mecanismo é, cada lado do octógono irá expandir ou retrair, de modo que se alguma das extremidades cruzar a parte interna do octógono, isso mostra que aquela extremidade é extremamente fraca e os desenvolvedores do projeto precisam melhorar aquela área. O \textit{Octalysis} foi a base de uma das avaliações realizadas com o jogo desenvolvido no âmbito deste trabalho. Esta avaliação será descrita na Seção~\ref{ssec:especialistas}.

%%%%%%%%%%%%%%%%%%%%%%%%%%%%%%%%%%%%%%%%%%%%%%%%%%%%%%%%%%%%%%%%%%%%%%%%%%%%%%%
%%%%%%%%%%%%%%%%%%%%%%%%%%%%%%%%%%%%%%%%%%%%%%%%%%%%%%%%%%%%%%%%%%%%%%%%%%%%%%%
%%%%%%%%%%%%%%%%%%%%%%%%%%%%%%%%%%%%%%%%%%%%%%%%%%%%%%%%%%%%%%%%%%%%%%%%%%%%%%%

\section{O Jogo Digital como Recurso para o Desenvolvimento do Pensamento Computacional} \label{sec:jogos_desenvolvimento_pc}

Na literatura referente ao Pensamento Computacional, é possível encontrar diversos exemplos de \textit{softwares} educativos que utilizam abordagens gamificadas e que podem auxiliar no desenvolvimento do Pensamento Computacional. Dentre esses exemplos, estão os ambientes gráficos de programação \textit{Scratch}, \textit{Alice}, \textit{Game Maker} e \textit{Kodu} e ferramentas de simulação e criação como o \textit{AgentSheets} e \textit{AgentCubes}~\cite{grover_computational_2013}. 

O \textit{Scratch}\footnote{\url{https://scratch.mit.edu/}}, sobre o qual já foram realizados diversos estudos de caso, conforme relatados por Oliveira et al~\cite{oliveira_ensino_2014}, Grover, Cooper e Pea~\cite{grover_assessing_2014} e Sáez-López, Román-González e Vázquez-Cano~\cite{saez-lopez_visual_2016}, é uma linguagem de programação e uma comunidade \textit{online}, desenvolvida pelo \acrshort{MIT}, através da qual crianças são introduzidas à programação por meio de blocos visuais. O \textit{Scratch} também inspirou a criação de outros softwares similares, como \textit{ScratchJr}\footnote{\url{https://www.scratchjr.org/}}, \textit{BeetleBlocks}\footnote{\url{http://beetleblocks.com/}} e \textit{Snap!}\footnote{\url{http://snap.berkeley.edu/}}.

\textit{Alice}, desenvolvido pela  \textit{Carnegie Mellon University}, é outro ambiente de programação baseado em blocos que permite a criação de animações e narrativas interativas e o desenvolvimento de jogos em \textit{3D}. De acordo com o \textit{website}\footnote{\url{http://www.alice.org/}} do projeto, ele foi criado com o objetivo de ser uma exposição inicial à programação orientada a objetos e desenvolver as habilidades do Pensamento Computacional por meio da exploração criativa. 

\textit{AgentSheets} e \textit{AgentCubes}, por sua vez, são ferramentas que oferecem uma interface do tipo “arraste e solte” e permitem a criação e compartilhamento de jogos e de simulações. Seus criadores acreditam que criar jogos, e não apenas jogá-los, ensina conceitos da Ciência da Computação, lógica e pensamento algorítmico\footnote{\url{http://www.agentsheets.com/index.html}}.

\textit{Softwares} educativos gamificados que funcionam como ambientes de desenvolvimento para crianças, como os citados anteriormente, costumam seguir o princípio do \textit{“low floor, high ceiling”}, que estabelece que deveria ser fácil para um iniciante utilizá-los (\textit{low floor}), mas que eles também devem ser poderosos o suficiente para satisfazer usuários mais avançados (\textit{high ceiling})~\cite{grover_computational_2013}.

Em 2016, Zanetti, Borges e Ricarte~\cite{zanetti_pensamento_2016} fazem uma revisão de trabalhos relacionados ao Pensamento Computacional, publicados em eventos brasileiros entre os anos de 2012 a 2015. Todavia, entre os 16 artigos adequados ao contexto da pesquisa, apenas seis trabalham o tema utilizando jogos digitais, dentre os quais apenas um era focado no Ensino Fundamental. Ainda, todos os jogos digitais avaliados nestes artigos trabalham o Pensamento Computacional no contexto de programação, característica que se repete em outros trabalhos. Portanto, no Capítulo~\ref{cap:jogo}, trazemos uma abordagem gamificada diferente para trabalhar o Pensamento Computacional, focando nas habilidades que podem ser desenvolvidas transversalmente no currículo do Ensino Fundamental I.
