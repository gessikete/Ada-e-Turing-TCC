A educação, desde 1988, é direito garantido a todos pela Constituição Federal e dever do Estado e da família. O texto constitucional determina no Capítulo III, artigo 205, que a educação deve ser “promovida e incentivada com a colaboração da sociedade, visando ao pleno desenvolvimento da pessoa, seu preparo para o exercício da cidadania e sua qualificação para o trabalho”. Sendo um processo contínuo, a educação se inicia nas primeiras etapas da vida e deve fornecer meios para que o sujeito possa progredir no trabalho e tenha o fomento necessário para prosseguir com estudos posteriores.

A obrigatoriedade e gratuidade do acesso à Educação Básica no Brasil é garantida dos quatro aos dezessete anos de idade, tanto aos cidadãos matriculados em idade regular quanto aos que não tiveram acesso a ela na idade adequada. A Lei n$^{\circ}$ 9.394~\cite{brasil_lei_1996}, Lei de Diretrizes e Bases da Educação Nacional, um dos documentos que norteiam a Educação Básica, determina que ela deve ser organizada em: Educação Infantil, Ensino Fundamental e Ensino Médio. Conforme a mesma lei, o desenvolvimento do sujeito é o objetivo final do conjunto dessas etapas, de forma que cada uma delas tem seu próprio escopo, trabalhando de forma crescente e complementar para:

\begin{itemize}
	\item Educação Infantil: o desenvolvimento integral da criança de até cinco anos, em seus aspectos físico, psicológico, intelectual e social, complementando a ação da família e da comunidade;
	\item Ensino Fundamental: a formação básica do cidadão - com duração de nove anos e iniciando-se aos seis anos de idade -, mediante:

	\begin{enumerate}
\item o desenvolvimento da capacidade de aprender, tendo como meios básicos o pleno domínio da leitura, da escrita e do cálculo;
\item a compreensão do ambiente natural e social, do sistema político, da tecnologia, das artes e dos valores em que se fundamenta a sociedade;
\item o desenvolvimento da capacidade de aprendizagem, tendo em vista a aquisição de conhecimentos e habilidades e a formação de atitudes e valores;
\item o fortalecimento dos vínculos de família, dos laços de solidariedade humana e de tolerância recíproca em que se assenta a vida social.
	\end{enumerate}

	\item Ensino Médio: consolidação da etapa final da educação básica, com duração mínima de três anos, terá como finalidades:

	\begin{enumerate}
		\item a consolidação e o aprofundamento dos conhecimentos adquiridos no Ensino Fundamental, possibilitando o prosseguimento dos estudos;
\item a preparação básica para o trabalho e a cidadania do educando, para continuar aprendendo, de modo a ser capaz de se adaptar com flexibilidade a novas condições de ocupação ou aperfeiçoamento posteriores;
\item o aprimoramento do educando como pessoa humana, incluindo a formação ética e o desenvolvimento da autonomia intelectual e do pensamento crítico;
\item a compreensão dos fundamentos científico-tecnológicos dos processos produtivos, relacionando a teoria com a prática, no ensino de cada disciplina.

	\end{enumerate}
\end{itemize}

A inclusão tanto de conceitos quanto de aparatos tecnológicos na Educação Básica respalda-se na utilização cotidiana destes recursos de forma natural por crianças, adolescentes e adultos. As novas mídias - devido à evolução e à disponibilidade tecnológica - são personalizadas e portáteis, o que, segundo Thorn~\cite{thorn_preschool_2008}, torna a exposição a elas possível não só em qualquer tempo ou circunstância, mas também de forma customizável. O autor ainda aponta que estudos realizados mostram que o desenvolvimento comportamental é impactado pela imersão de crianças em mídias e, além disso, também auxiliaria no aprendizado.

Em 1999, Valente~\cite{valente_informatica_1999} descreve uma visão histórica da evolução de perspectivas pedagógicas através de um apanhado dos avanços alcançados nos Estados Unidos e na França justamente devido à Informática. O autor ainda ressalta que estas transformações pedagógicas estariam aquém do desejado, pois apesar dos grandes avanços ao acesso de computadores nessas escolas, a abordagem educacional continuava sendo predominantemente tradicional.

No Brasil, as políticas e propostas pedagógicas de Informática na Educação foram respaldadas nas pesquisas realizadas entre as universidades e as escolas públicas, ao contrário dos Estados Unidos - onde o uso descentralizado e independente das decisões governamentais foi pressionado pelo desenvolvimento tecnológico, necessidade de profissionais qualificados e competição pelo livre mercado das empresas que produzem os \textit{softwares} das universidades e escolas - e da França - primeiro país ocidental que se preparou para o desafio da Informática na Educação, tornando-se modelo para o mundo, de forma que as mudanças foram centralizadas pelo governo na produção do \textit{hardware} e do \textit{software}, assim como na formação das novas gerações para o domínio e produção dessas tecnologias~\cite{valente_informatica_1999}. 

Valente~\cite{valente_integracao_2016}, em seus estudos sobre implantação tecnológica na educação, identifica diferentes estratégias adotadas por diferentes países, de modo que essas podem ser classificadas em três grandes categorias: a inclusão de assuntos da Ciência da Computação, em especial a programação, fora da sala de aula e no currículo; inclusão de disciplinas no currículo que exploram o Pensamento Computacional por meio de diferentes atividades, como jogos, robótica, além da exploração de seus conceitos de maneira transversal em diferentes disciplinas do currículo; e a implantação de atividades computacionais no currículo. Questões como a preparação do professor para desenvolver essas atividades e como avaliar o aluno são abordadas em diversos trabalhos.

Aqui, destacamos o papel essencial do Ensino Fundamental como fase de formação básica do cidadão, em especial quanto à inclusão tecnológica. O Ensino Fundamental de nove anos, de acordo com o Ministério da Educação~\cite{ministerio_da_educacao_ensino_2009}, está subdividido do $1^o$ ao $5^o$ ano como Anos Iniciais, enquanto o $6^o$ ao $9^o$ são os Anos Finais, respectivamente, Ensino Fundamental I e Ensino Fundamental II e neles, além da alfabetização e letramento, o pleno domínio do cálculo e a compreensão da tecnologia são pontos evidenciados na lei.

Apesar dos vários estudos e das leis apresentadas, em pesquisa realizada em 2016 pelo \acrfull{CETIC.br}, menos da metade dos alunos de $4^o$ e $5^o$ ano do Ensino Fundamental recebem orientação dos professores para uso da internet. Ainda assim, é mostrado que 75\% e 80\% dos alunos nesse período escolar utilizam a internet para aprender coisas novas procurando por informações no Google ou em outro buscador e assistindo vídeos, respectivamente. Além disso, mais da metade desses alunos acham que a aula fica mais legal, aprendem mais ou que prestam mais atenção quando o professor usa a internet.

É notável a necessidade da inclusão da Tecnologia na Educação tanto como meio de preparar o cidadão para as necessidades de uma sociedade cada vez mais dependente da tecnologia, quanto pela percepção dos próprios alunos que vivenciam a tecnologia no dia a dia. Para tanto, exploraremos no Capítulo~\ref{cap:pensamento} o Pensamento Computacional, um dos meios apontados por Valente para a implantação tecnológica na educação.