Conforme discorrido ao longo deste trabalho, o Pensamento Computacional envolve a decomposição de problemas em subproblemas mais simples por meio de algoritmos e a capacidade de aplicar essas soluções em problemas similares, de forma que essas soluções podem ser apresentadas em um formato computacional ou não. Essas habilidades, que se tornam pré-requisito em uma sociedade cada vez mais dependente da tecnologia devem, idealmente, ser desenvolvidas desde os primeiros anos do ensino, em destaque o Ensino Fundamental, período escolar que, por lei, é responsável pela formação básica do cidadão, especialmente quanto à inclusão tecnológica.

Além da influência direta do Pensamento Computacional no desempenho do indivíduo quanto à inclusão tecnológica, sua importância também é destaque quanto à formação do aluno. A influência no desempenho de outras disciplinas, como matemática e física, evidencia-se quanto ao desenvolvimento do raciocínio lógico, decomposição de problemas, algoritmos, abstração, etc.

Apesar de sua crescente importância, não existem muitas pesquisas acerca do Pensamento Computacional no contexto do Ensino Fundamental I. Ademais, prevalecem as propostas e estudos que têm como objetivo promovê-lo por meio do ensino de programação, o que, embora seja uma estratégia válida, nem sempre será a mais adequada, além de perpetuar a preconcepção de que o Pensamento Computacional é um resultado direto da prática de programação. 

Diante da relevância do tema, este trabalho teve como objetivo geral apoiar o desenvolvimento de habilidades do Pensamento Computacional em alunos do Ensino Fundamental I. Para tanto, traçamos cinco objetivos específicos que foram abordados ao longo dos estudos: 

\begin{itemize}
	\item apropriação da temática do Pensamento Computacional;
	\item levantamento de usos dos jogos digitais no desenvolvimento do Pensamento Computacional;
	\item proposição de protótipo de jogo como apoio ao desenvolvimento de habilidades do Pensamento Computacional;
	\item realização da avaliação do jogo desenvolvido, visando identificar possíveis manutenções na ferramenta;
	\item realização de testes com especialistas na área e com possíveis usuários do jogo.
\end{itemize}

Através de pesquisa exploratória acerca do tema, com base no estudo de diversas abordagens e definições, foram construídos os sustentáculos do trabalho, o que proporcionou a apropriação da temática do Pensamento Computacional. Partindo desse ponto, o levantamento de usos de jogos digitais como apoio ao desenvolvimento do conteúdo foi realizado da mesma maneira, por meio da averiguação de estudos existentes e da constatação da necessidade de criação de um jogo digital educacional como ferramenta que pudesse demonstrar como as habilidades do Pensamento Computacional podem ser inseridas de forma lúdica e transversal no Ensino Fundamental I sem o uso de programação. 
 
Para tanto, foi proposto o protótipo de jogo como apoio ao desenvolvimento de habilidades do Pensamento Computacional. O jogo \textbf{As Aventuras de Ada e Turing} buscou, em cada uma das suas fases, promover habilidades do Pensamento Computacional através de atividades que estavam inseridas dentro de uma narrativa simples, mas que, juntamente com os elementos visuais, procurava engajar o usuário a concluir as dinâmicas propostas. Dentre as habilidades que se pretendia promover, estão: uso de vocabulário, desenho de soluções por meio de análise e de criação de algoritmos e realização de testes e depuração. 

Para aferir a usabilidade do protótipo, foram realizados testes com especialistas e com o público-alvo da ferramenta. Embora precisem ser ampliados, os testes realizados possibilitaram a identificação de vários aspectos de usabilidade que podem ser melhorados para desenvolver um jogo que se aproxime mais das necessidades e limitações do usuário de forma a criar uma interface transparente, que não demande esforço, por parte do jogador, para seu entendimento e utilização.

Os testes efetuados ainda apontam a importância da existência de uma equipe multidisciplinar para a criação de \textit{software} educacionais, ou seja, um grupo de trabalho composto por pedagogo (criação do conteúdo didático adequado), \textit{designer} (criação da arte do jogo e planejamento da disposição dos elementos em tela), psicólogo (aplicação de avaliações que validem o que o jogo se propõe a desenvolver) e desenvolvedores (codificação do jogo propriamente dito). A junção desses componentes agrega o conhecimento processual fundamental para a criação de atividades que, de fato, promovam o desenvolvimento do Pensamento Computacional.

Consideramos que o jogo se mostrou uma abordagem plausível quanto à aplicação das habilidades que compõem o Pensamento Computacional em uma ferramenta de apoio, prática que favorece o desenvolvimento do tema mesmo em contextos onde o tema não está incluído no currículo. Ademais, cria subsídios para novas pesquisas que utilizem essa e outras formas de desenvolver o Pensamento Computacional. 

Dentre as possíveis melhorias e trabalhos futuros, além dos testes de usabilidade, é necessário que sejam realizados testes com amostras maiores e com o intuito de validar a eficácia da ferramenta como forma de promover o desenvolvimento das habilidades do Pensamento Computacional (testes cognitivos). Esses testes não foram realizados uma vez que era necessária a efetuação prévia dos testes de usabilidade para alcançar uma interface mais adequada ao público-alvo, além do fato de que tais testes exigiriam acompanhamentos anteriores e posteriores dos indivíduos por profissionais capacitados em Ciências Cognitivas para verificar a aquisição efetiva das habilidades propostas, o que reflete a importância de uma equipe multidisciplinar. 

Além de testes mais aprofundados, também deve ser realizada uma reformulação do tutorial, para que este se torne mais interativo e possibilite um treinamento guiado mais extenso. Ademais, a ampliação do jogo para apoiar outras habilidades do Pensamento Computacional, como resolver problemas em grupo, utilizando uma abordagem onde a criança só poderá completar uma atividade com a ajuda de outro personagem do jogo ou de um colega. Também devem ser criados níveis para a diferenciação entre as idades das crianças, e a implementação de um mecanismo de registro do percurso do aluno, de forma a gerar um \textit{feedback} para o professor, de maneira que este possa orientar os alunos nos tópicos que apresentarem mais dificuldades.