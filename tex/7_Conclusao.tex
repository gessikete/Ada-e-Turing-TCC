Conforme discorrido ao longo deste trabalho, o Pensamento Computacional envolve a decomposição de problemas em subproblemas mais simples por meio de algoritmos e a capacidade de aplicar essas soluções em problemas similares, além de determinar se o uso do computador pode auxiliar na resolução da questão. Essas habilidades, que se tornam pré-requisito em uma sociedade cada vez mais dependente da tecnologia, devem, idealmente, ser desenvolvidas desde os primeiros anos do Ensino Fundamental, período escolar responsável pela formação básica do cidadão. 

Apesar de sua crescente importância, não existem muitas pesquisas acerca do Pensamento Computacional no contexto do Ensino Fundamental I. Ademais, prevalecem as propostas e estudos que têm como objetivo promovê-lo por meio do ensino de programação, o que, embora seja uma estratégia válida, nem sempre será a mais adequada, além de perpetuar a preconcepção de que o Pensamento Computacional é um resultado direto da prática de programação. 

Haja vista a relevância do Pensamento Computacional, este trabalho teve como objetivo geral apoiar o desenvolvimento das habilidades que o constituem em alunos do Ensino Fundamental I, para tanto traçamos cinco objetivos específicos que foram abordados ao longo dos estudos: 

\begin{itemize}
	\item apropriação da temática do Pensamento Computacional;
	\item levantamento de usos dos jogos digitais no desenvolvimento do Pensamento Computacional;
	\item propor protótipo de jogo como apoio ao desenvolvimento de habilidades do Pensamento Computacional;
	\item realizar avaliação do jogo desenvolvido, visando identificar possíveis manutenções na ferramenta;
	\item realizar testes com especialistas na área e com possíveis usuários do jogo.
\end{itemize}

Através de pesquisa exploratória acerca do tema, com base no estudo de diversas abordagens e definições, foram construídos os sustentáculos do trabalho, o que proporcionou a apropriação da temática do Pensamento Computacional. Partindo desse ponto, o levantamento de usos de jogos digitais como apoio ao desenvolvimento do conteúdo foi realizado da mesma maneira, por meio da averiguação de estudos existentes e constatação da necessidade de criação de um jogo digital educacional como ferramenta que pudesse demonstrar como as habilidades do Pensamento Computacional podem ser inseridas de forma lúdica e transversal no currículo atual do Ensino Fundamental I sem o uso de programação. 
 
Para tanto, foi proposto o protótipo de jogo como apoio ao desenvolvimento de habilidades do Pensamento Computacional. “\textbf{As Aventuras de Ada e Turing}” buscou, em cada uma das suas fases, promover habilidades do Pensamento Computacional através de atividades que estavam inseridas dentro de uma narrativa simples, mas que, juntamente com os elementos visuais, procurava engajar o usuário a concluir as dinâmicas propostas. Dentre as habilidades que se pretendia promover, estão: uso de vocabulário, desenho de soluções por meio de análise e de criação de algoritmos e realização de testes e depuração. 

Para aferir a usabilidade do protótipo, foram realizados testes com especialistas e com o público-alvo da ferramenta. Embora não tenha sido possível tirar conclusões precisas, os testes realizados possibilitaram a identificação de vários aspectos de usabilidade que podem ser melhorados para desenvolver um jogo que se aproxime mais das necessidades e limitações do usuário de forma a criar uma interface transparente, que não demande esforço, por parte do jogador, para seu entendimento e utilização. 

Além desses testes de usabilidade, é necessário que sejam realizados testes com amostras maiores e com o intuito de validar a eficácia da ferramenta como forma de promover o desenvolvimento das habilidades do Pensamento Computacional. Também deve ser realizada uma reformulação do tutorial, para que este se torne mais interativo e possibilite um treinamento guiado mais extenso, a criação de níveis para a diferenciação entre as idades das crianças, e implementação de um mecanismo de registro do percurso do aluno, de forma a gerar um \textit{feedback} para o professor, de maneira que este possa orientar os alunos nos tópicos que apresentarem mais dificuldades.
