Este trabalho consiste no estudo relacionado ao Pensamento Computacional, tema que foi apresentado há algum tempo e, atualmente, está em crescente ascensão e destaque, e seu desenvolvimento desde as séries inicias do Ensino Fundamental. A partir da observação da relevância do tópico e da escassez de iniciativas, principalmente no referido nível, com o intuito de promover as habilidades que o compõem, verificou-se a necessidade da criação de uma ferramenta de apoio ao desenvolvimento do Pensamento Computacional voltada ao Ensino Fundamental I. Para tanto, foi realizada pesquisa exploratória visando a apropriação do conteúdo e das formas como ele é abordado na bibliografia, após foram realizados o desenvolvimento do jogo digital “\textbf{As Aventuras de Ada e Turing}” e avaliações acerca da sua usabilidade. O jogo digital foi escolhido com o objetivo de demonstrar uma proposta de emprego lúdico e transversal das habilidades que constituem o Pensamento Computacional ao currículo já existente. Com este trabalho, espera-se demonstrar a possibilidade de criação de uma ferramenta que favoreça o desenvolvimento das habilidades do Pensamento Computacional, além de incentivar mais pesquisas a respeito do assunto.