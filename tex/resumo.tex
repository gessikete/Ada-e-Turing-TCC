O uso de computadores para as mais diversas tarefas é uma realidade de nosso tempos. Dessa forma, saber utilizar o computador para tarefas rotineiras (e.g. enviar e-mail ou acessar redes sociais) não será uma habilidade suficiente para as pessoas em um futuro próximo. Neste sentido, alguns países estão investindo atualmente em políticas educacionais para inserir programação e conteúdos que promovam o Pensamento Computacional na grade curricular nos primeiros anos escolares. O Pensamento Computacional pode ser definido como um conjunto de habilidades que nos possibilita dividir problemas complexos em subproblemas menores por meio de uma sequência de passos, revisar como a solução pode ser aplicada em problemas similares e, por fim, determinar se um computador pode ajudar na resolução do problema. Infelizmente, são ainda poucas as iniciativas no Brasil destinadas a promover as habilidades do Pensamento Computacional  no Ensino Fundamental I. Aqui, apresentamos o jogo digital “\textbf{As Aventuras de Ada e Turing}” e as avaliações acerca da sua usabilidade. O jogo digital foi escolhido com o intuito de demonstrar uma proposta de emprego lúdico e transversal das habilidades que constituem o Pensamento Computacional ao currículo já existente. Com este trabalho, espera-se demonstrar a possibilidade de criação de uma ferramenta que favoreça o desenvolvimento das habilidades do Pensamento Computacional, além de incentivar pesquisas futuras sobre este tema.