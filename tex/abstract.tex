As the dependence of simple tasks on the use of computers increases, and as this use in itself becomes more complex by the day, some nations have started to invest on educational policies that insert programming and Computational Thinking on the curriculum of elementary schools. Computational Thinking is the set of abilities with which one divides complex problems into smaller subproblems, establishes in which way a given solution can be applied to different problems and decides whether a computer could help with the problem’s resolution. Unfortunately, Brazil’s elementary school system lacks initiatives on teaching these abilities. The present work presents the digital game “\textbf{The Adventures of Ada and Turing}” and the conclusions as to its usability. The game was selected for it represents a playful and transverse application of Computational Thinking to the curriculum of brazilian Elementary Schools already in place. Apart from promoting further research on the field, we intend to pose a possible tool to favor the development of Computational Thinking abilities.